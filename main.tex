\documentclass[a4paper, 12pt]{article}

\usepackage{config}

\begin{document}
	\noindent\textbf{Renan da Silva Guedes | 223979}
	
	\vspace{.5cm}
	O \textit{paper} analisado tratou do estudo referente ao posicionamento de entradas e saídas de ar (exaustores) em instalações destinadas a comportar suínos. Nele, pôde-se ver a importância da dinâmica dos fluidos computacional no dimensionamento dos estabelecimentos e como as diferentes configurações de posição dos mecanismos de fluxo do ar interferem na troca de calor com o ambiente. Dessa forma, com base no estudo, notou-se a busca por garantir condições propicias para o desenvolvimento dos animais a partir da manutenibilidade das suas necessidades vitais. Com base nisso, a redução das taxas de mortalidade e estresse enfrentados no ambiente onde são alojados torna-se viável.  
	
	Para a realização da análise foi feito uso de métodos computacionais, onde o estudo das condições de contorno foram essenciais de modo a restringir as variáveis da questão e apontar dados pertinentes a partir da boa seletividade. Dessa forma, além do posicionamento das entradas e saídas de ar, foi feito o estudo dos parâmetros internos potenciais à variação das condições de fluxo e temperatura do ar. Sendo assim, fazendo uso de sensores anemômetros foi realizada a medição da velocidade do ar em diferentes pontos (30 pontos diferentes) e estabelecida média diária. Para as paredes foi feito uso de técnicas não destrutivas que incorporaram termômetros infravermelhos e imagens termográficas que aferem a temperatura ao longo da área e retornam os dados como um gradiente térmico em diferentes tonalidades.
	
	Os artifícios aplicados no decorrer desse projeto são de fundamental importância e possuem bastantes aplicações em áreas ligadas à agricultura. Por exemplo, nas casas de vegetação são vistas condições semelhantes no que diz respeito a regulação do ar no interior do recinto. Desta maneira, ao produzir plantas em ambiente protegido, é necessário o dimensionamento de entradas de ar e exaustores de modo a permitir as trocas gasosas entre o interior e exterior da estrutura, viabilizando a renovação do ar internamente. 
	
	Estudos demonstram que é aconselhável a utilização de exaustores na parte superior das casas de vegetação, tendo em vista a convecção do ar quente para camadas superiores. Ou seja, análogo ao que é visto nas instalações para suínos, procedimentos semelhantes podem ser adotados em menor escala como é visto nas \textit{green houses}, porém as condições devem ser adaptadas segundo as solicitações da cultura desejada.
	
	Dessa forma, a partir de todo o sensoriamento remoto realizado, uma abordagem pertinente de ser empregada consiste na criação de uma API (\textit{Application Programming Interface}) que lide com os dados recebidos pelo sistema de aquisição sendo, dessa forma, incorporada à base de um \textit{software} de aplicação. Comumente esse termo é empregado no desenvolvimento de aplicações \textit{web} multiplataforma, tendo em vista as constantes mudanças e sofisticações sendo incrementadas no setor. 
	
	Com base nos diferentes métodos presentes no protocolo HTTP, por exemplo, é possível dar funcionalidades a uma aplicação que permitem ao usuário o acesso aos parâmetros aferidos em tempo real e sugira mudanças com base no panorama presente no momento. Os métodos citados anteriormente dizem respeito aos \textit{controllers} de uma aplicação, ou seja, representam funções que viabilizam o chamado CRUD (\textit{Create}, \textit{Read}, \textit{Update} e \textit{Delete}) a partir das diferentes rotas acessadas. 
	
	Nessa caso, a proposta da API contemplaria somente o \textit{backend} da aplicação, podendo ser aplicado o conceito de SPA (\textit{Single Page Application}) no é viabilizada maior modularização do lado servidor e o eventual uso com diferentes \textit{frameworks} do mercado. Tendo em vista esses conceitos, torna-se possível criar uma aplicação que gerencie dados captados por sistemas de aquisição associados a sensores presentes em \textit{green houses} e permita a tomada de decisão com base nos recursos presentes nas rotas, sendo que há maior flexibilidade para o acesso à informação, seja via \textit{web}, \textit{mobile} ou \textit{desktop}.
\end{document}